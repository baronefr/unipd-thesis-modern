% -- CODE SNIPPETS --
\RequirePackage{listings}

% algoritms
\RequirePackage{algorithm}
\RequirePackage{algorithmic}

% Colors for code snippets highlights
\definecolor{commentsColor}{RGB}{36, 161, 156}
\definecolor{numColor}{RGB}{71, 96, 114}
\definecolor{stringColor}{RGB}{205, 63, 62}
\definecolor{kwColor}{RGB}{248, 6, 204}
\definecolor{bgCodeColor}{RGB}{255, 249, 249}

% Style of code snippets
\lstdefinestyle{codeSnippet} {
  backgroundcolor=\color{bgCodeColor},
  commentstyle=\color{commentsColor},
  keywordstyle=\color{kwColor},
  numberstyle=\tiny\color{numColor},
  stringstyle=\color{stringColor},
  basicstyle=\ttfamily\footnotesize,
  breakatwhitespace=false,         
  breaklines=true,                 
  captionpos=b,                    
  keepspaces=true,                 
  numbers=left,                    
  numbersep=5pt,                  
  showspaces=false,                
  showstringspaces=false,
  showtabs=false,                  
  tabsize=2
}

\lstset{style=codeSnippet}





% -- BOXES --
% inspired from https://tex.stackexchange.com/questions/698167/how-to-make-the-inner-icon-solid-in-fontawesome
\RequirePackage[framemethod=tikz]{mdframed}
\RequirePackage[fixed]{fontawesome5}


% warning
\mdfdefinestyle{warning}{
	topline=false, bottomline=false,
	leftline=false, rightline=false,
	nobreak,
	singleextra={%
        \coordinate (TL) at (P-|O);
        \node at(TL) {\color{black}\faExclamationTriangle};
		\draw[very thick](P-|O)++(0,-0.8em)--(O);
	}
}

\newenvironment{warn}[1][Warning:]{
	\medskip
	\begin{mdframed}[style=warning]
		\noindent{\textbf{#1}}
}{
	\end{mdframed}
}

% info
\mdfdefinestyle{info}{%
	topline=false, bottomline=false,
	leftline=false, rightline=false,
	nobreak, %backgroundcolor=black!10,
	singleextra={%
        \coordinate (TL) at (P-|O);
        \node at(TL) {\color{black}\faInfoCircle};
		\draw[very thick](P-|O)++(0,-0.8em)--(O);
	}
}

\newenvironment{info}[1][Info:]{
	\medskip
	\begin{mdframed}[style=info]
		\noindent{\textbf{#1}}
}{
	\end{mdframed}
}


% summary
\mdfdefinestyle{summary}{%
    topline=false, bottomline=false,
    leftline=false, rightline=false,
    nobreak,
    backgroundcolor=pastelgreen!10,
    singleextra={%
        \coordinate (TL) at (P-|O);
        \node[shape=circle, fill, color=white] at (TL) {};
        \node at(TL){\color{pastelgreen}\faInfoCircle};
        \draw[very thick, pastelgreen](P-|O)++(0,-0.3em)--(O);
        }       
}

\newenvironment{summary}[1][Summary:]{ 
    \medskip
    \begin{mdframed}[style=summary]
        \noindent{\textbf{#1}}
}{
    \end{mdframed}
}

% example
\mdfdefinestyle{example}{%
    topline=false, bottomline=false,
    leftline=false, rightline=false,
    nobreak,
    skipabove=0pt, skipbelow=0pt,
    singleextra={%
        \coordinate (TL) at (P-|O);
        \node[shape=circle, fill, color=white] at (TL) {};
        \node at(TL){\footnotesize\faPlusCircle};
        \draw[very thick, black](P-|O)++(0.05em,-0.4em)--(0.05em,0em);
        }       
}

\newenvironment{example}[1][Example:]{ 
    \medskip
    \begin{mdframed}[style=example]
        \noindent{\textbf{#1}}
}{
    \end{mdframed}
}


\RequirePackage[most]{tcolorbox}


\tikzset{vertex style/.style={
    draw=#1,
    thick,
    fill=#1!70,
    text=white,
    ellipse,
    minimum width=2cm,
    minimum height=0.75cm,
    font=\small,
    outer sep=3pt,
  },
  text style/.style={
    sloped,
    text=black,
    font=\footnotesize,
    above
  }
}


% -- EQUATIONS --
\RequirePackage{xparse}

% overarrows
%   from https://tex.stackexchange.com/questions/8720/overbrace-underbrace-but-with-an-arrow-instead
\NewDocumentCommand{\overarrow}{O{=} O{\uparrow} m}{%
  \overset{\makebox[0pt]{\begin{tabular}{@{}c@{}}#3\\[0pt]\ensuremath{#2}\end{tabular}}}{#1}
}
\NewDocumentCommand{\underarrow}{O{=} O{\downarrow} m}{%
  \underset{\makebox[0pt]{\begin{tabular}{@{}c@{}}\ensuremath{#2}\\[0pt]#3\end{tabular}}}{#1}
}

\RequirePackage{accents}
%\renewcommand{\undertilde}[1]{\underaccent{\tilde}{#1}}

\DeclareMathOperator{\Tr}{Tr}

\newcommand{\bigequalexpl}[1]{%
  \underset{\substack{\uparrow\\\\\mathrlap{\text{\hspace{-5em}#1}}}}{=}}



% -- GITHUB LINK --

\newcommand{\prettygitlink}[1]{
    \begin{table}[H]
    \centering
    \renewcommand{\arraystretch}{1.2}
        \begin{tabular}{cl}
        \hline
        {\large\faGithub} & \href{https://github.com/#1}{#1} \\ \hline
        \end{tabular}
    \end{table}
}



% -- extra packages --
\RequirePackage{quantikz} %*
\RequirePackage{annotate-equations} %*
\RequirePackage{tabularray} %*
\RequirePackage{makecell} %*
\RequirePackage{afterpage} %*
\RequirePackage[export]{adjustbox} %*
\RequirePackage{leftindex} %*


